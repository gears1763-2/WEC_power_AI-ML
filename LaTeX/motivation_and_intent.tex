\chapter{Motivation and Intent}

Wave energy converter (WEC) technology is an emerging field within the domain of renewable energy, especially for regions with a high wave energy potential such as Vancouver Island. However, compared to other renewable energy technologies such as solar and wind, WEC dynamics are highly complex and expensive to model and simulate, and so methods of reducing this computational expense (without sacrificing an excessive amount of detail) are desirable. This is precisely the intent of this work \par 
To that end, the approach taken in this work is inspired by the historical successes of the so-called perturbation methods. Essentially, this is the technique of obtaining a solution (albeit approximate) to a complex dynamics problem by first obtaining an exact solution to a related, reduced problem, and then seeking an appropriate correction (or \textit{perturbation}) of this exact solution to better approximate the true, complex dynamics.\footnote{This could also be a sequence of corrections, rather than simply one.} The canonical example of the power of these methods are their contribution to the discovery of the planet Neptune \cite{Bogolyubov_2024}:\par
\vspace{4mm}
\begin{small}
\textit{Perturbation theory has been investigated by prominent mathematicians (e.g., Laplace, Poisson, Gauss), and as such the computations can be performed with very high accuracy. For example, the existence of the planet Neptune was postulated in 1848 by mathematicians J.C. Adams and U. le Verrier, with this postulation being based on the deviations (i.e. perturbations) in the observed motion of the planet Uranus. The existence of Neptune was then confirmed by way of observation by the astronomer J.G. Galle (who received coordinates from le Verrier). This represented a triumph of perturbation theory.}
\end{small}\par 
\vspace{4mm}
\noindent The approach taken in this work can itemized as follows

\begin{enumerate}
	\item Define an appropriate ``related, reduced problem" for WEC technology.
	\item Seek an exact solution to the reduced problem.
	\item Identify an appropriate form for the perturbation.
	\item Generate data suitable for use in mining and artificial intelligence / machine learning (AI/ML).
	\item Mine the data for some initial insight into the nature of the perturbation.
	\item Train an AI/ML model to serve as the perturbation.
\end{enumerate}